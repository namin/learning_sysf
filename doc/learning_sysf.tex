% For review submission
%\documentclass[acmsmall,review,anonymous]{acmart}
\documentclass[acmsmall]{acmart}

\settopmatter{printfolios=true,printccs=false,printacmref=false}
%% For double-blind review submission, w/ CCS and ACM Reference
%\documentclass[acmsmall,review,anonymous]{acmart}\settopmatter{printfolios=true}
%% For single-blind review submission, w/o CCS and ACM Reference (max submission space)
%\documentclass[acmsmall,review]{acmart}\settopmatter{printfolios=true,printccs=false,printacmref=false}
%% For single-blind review submission, w/ CCS and ACM Reference
%\documentclass[acmsmall,review]{acmart}\settopmatter{printfolios=true}
%% For final camera-ready submission, w/ required CCS and ACM Reference
%\documentclass[acmsmall]{acmart}\settopmatter{}


%% Journal information
%% Supplied to authors by publisher for camera-ready submission;
%% use defaults for review submission.
\acmJournal{PACMPL}
\acmVolume{1}
\acmNumber{CONF} % CONF = POPL or ICFP or OOPSLA
\acmArticle{1}
\acmYear{2018}
\acmMonth{1}
\acmDOI{} % \acmDOI{10.1145/nnnnnnn.nnnnnnn}
\startPage{1}

%% Copyright information
%% Supplied to authors (based on authors' rights management selection;
%% see authors.acm.org) by publisher for camera-ready submission;
%% use 'none' for review submission.
\setcopyright{none}
%\setcopyright{acmcopyright}
%\setcopyright{acmlicensed}
%\setcopyright{rightsretained}
%\copyrightyear{2018}           %% If different from \acmYear

%% Bibliography style
\bibliographystyle{ACM-Reference-Format}
%% Citation style
%% Note: author/year citations are required for papers published as an
%% issue of PACMPL.
\citestyle{acmauthoryear}   %% For author/year citations


%%%%%%%%%%%%%%%%%%%%%%%%%%%%%%%%%%%%%%%%%%%%%%%%%%%%%%%%%%%%%%%%%%%%%%
%% Note: Authors migrating a paper from PACMPL format to traditional
%% SIGPLAN proceedings format must update the '\documentclass' and
%% topmatter commands above; see 'acmart-sigplanproc-template.tex'.
%%%%%%%%%%%%%%%%%%%%%%%%%%%%%%%%%%%%%%%%%%%%%%%%%%%%%%%%%%%%%%%%%%%%%%


%% Some recommended packages.
\usepackage{booktabs}   %% For formal tables:
                        %% http://ctan.org/pkg/booktabs
\usepackage{subcaption} %% For complex figures with subfigures/subcaptions
                        %% http://ctan.org/pkg/subcaption


\begin{document}

%% Title information
\title[Short Title]{Learning in System F}


%% Author information
%% Contents and number of authors suppressed with 'anonymous'.
%% Each author should be introduced by \author, followed by
%% \authornote (optional), \orcid (optional), \affiliation, and
%% \email.
%% An author may have multiple affiliations and/or emails; repeat the
%% appropriate command.
%% Many elements are not rendered, but should be provided for metadata
%% extraction tools.

%% Author with single affiliation.
\author{Joey Velez-Ginorio}
\affiliation{
  \department{Brain \& Cognitive Sciences}
  \institution{Massachusetts Institute of Technology}
  \streetaddress{43 Vassar St.}
  \city{Cambridge}
  \state{Massachusetts}
  \postcode{02139}
  \country{U.S.A.}                    %% \country is recommended
}
\email{joeyv@mit.edu}   

%% Author with single affiliation.
\author{Nada Amin}
\affiliation{
  \department{John A. Paulson School of Engineering and Applied Sciences}
  \institution{Harvard University}
  \streetaddress{29 Oxford St.}
  \city{Cambridge}
  \state{Massachusetts}
  \postcode{02138}
  \country{U.S.A.}                    %% \country is recommended
}
\email{namin@seas.harvard.edu}


%% Abstract
%% Note: \begin{abstract}...\end{abstract} environment must come
%% before \maketitle command
\begin{abstract}
Program synthesis, type inhabitance, inductive programming, and theorem proving. Different names for the same problem: learning programs from data. Sometimes the programs are proofs, sometimes they’re terms. Sometimes data are examples, and sometimes they’re types. Yet the aim is the same. We want to construct a program which satisfies some data. We want to learn a program.

What might a programming language look like, if its programs could also be learned? We give it data, and it learns a program from it. This work shows that System F yields a simple approach for learning from types and examples. Beyond simplicity, System F gives us a guarantee on the soundness and completeness of learning. We learn correct programs, and can learn all observationally distinct programs in System F. Unlike previous works, we don't restrict what examples can be. As a result, we show how to learn arbitrary higher-order programs in System F from types and examples. 

\end{abstract}


%% Keywords
%% comma separated list
\keywords{Program Synthesis, Type Theory, Inductive Programming}  %% \keywords are mandatory in final camera-ready submission


%% \maketitle
%% Note: \maketitle command must come after title commands, author
%% commands, abstract environment, Computing Classification System
%% environment and commands, and keywords command.
\maketitle


\section{Introduction}

Imagine we're teaching you a program. Your only data is the type $nat \!\to\! nat$. It takes a natural number, and returns a natural number. Any ideas? Perhaps a program which computes...
$$f(x) = x, \;\;\;\;\;\;f(x) = x + 1,\;\;\;\;\;\; f(x) = x + 2,\;\;\;\;\;\; f(x) = x + \cdots$$
The good news is that $f(x) = x + 1$ is correct. The bad news is that the data let you learn a slew of other programs too. It doesn't constrain learning enough if we want to teach $f(x) = x + 1$. As teachers, we can provide better data.

Round 2. Imagine we're teaching you a program. But this time we give you an example of the program's behavior. Your data are the type $nat \!\to\! nat$ and an example $f(1) = 2$. It takes a natural number, and returns the successor. Any ideas? Perhaps a program which computes...
$$f(x) = x + 1,\;\;\;\;\;\; f(x) = x + 2 - 1,\;\;\;\;\;\; f(x) = x + 3 - 2,\;\;\;\;\;\;\cdots$$
The good news is that $f(x) = x + 1$ is correct. And so are all the other programs, as long as we're agnostic to some details. Types and examples impose useful constraints on learning. It's the data we use when learning in System F \cite{girard1989proofs}.

Existing work can learn successor from similar data \cite{osera2015program, polikarpova2016program}. But suppose $nat$ is a church encoding. For some base type $A$, $nat \coloneqq (A \to A) \to (A \to A)$. Natural numbers are then higher-order functions. They take and return functions. Suddenly, existing work can no longer learn successor. 

The difficulty is with how to handle functions in the return type. The type $nat \!\to\! nat$ returns a function, a program of type $nat$. To learn correct programs, you need to ensure candidates are the correct type or that they obey examples. Imagine we want to verify that our candidate program $f$ obeys $f(1)=2$. With the church encoding, $f(1)$ is a function, and so is $2$. In other words, $f(1)=2$ requires that we decide the equivalence of functions---which is undecidable in a Turing-complete language \cite{sipser2006introduction}. Functions in the return type create this issue. There are two ways out:

\begin{enumerate}
\item Don't allow functions in the return type, keep Turing-completeness.
\item Allow functions in the return type, leave Turing-completeness.
\end{enumerate}


% state issue with functions in return type
% mention two possible solutions: don't allow them, focus on rest of 
% program space.... or our solution. keep them, use an expressive but
% normalizing language as your languyage for learning.


% Ok, that's surprising but who cares? Mention interesting higher-order.
% Why can't we do it though? Essentially the undecidability of program equality for expressive-languages.
% What's our solution in a nutshell. Lets' pick a strongly normalizing calculus which is still expressive enough for programs we care about.
% State contributions explicitly, bulleted:
% -- Learn arbitrary higher-order programs in System F.
% -- Prove soundness and completeness of learning
% -- Implement learning, extending strong theoretical guarantees in practice.


% present SysF syntax/typechecking/eval at end
\section{Learning from Types}
% present the augmented relation
% state completeness/soundness result? or prove...
\section{Learning from Examples}
% present the augmented relation
% state completeness/soundness result

\section{Implementation}
% how to make combinatorial search easier
% - enumerating normal form programs
% - deducing operationally distinct type applications
% - memoizing recurisve calls?
% - algebraic data types let us use examples productively
%    - sum types split examples
%	 - product types generate subproblems

\section{Experiments}
\section{Related Work}
\section{Conclusion}


%% Acknowledgments
%%\begin{acks}                            %% acks environment is optional
                                        %% contents suppressed with 'anonymous'
  %% Commands \grantsponsor{<sponsorID>}{<name>}{<url>} and
  %% \grantnum[<url>]{<sponsorID>}{<number>} should be used to
  %% acknowledge financial support and will be used by metadata
  %% extraction tools.
%%  This material is based upon work supported by the
%%  \grantsponsor{GS100000001}{National Science
%%    Foundation}{http://dx.doi.org/10.13039/100000001} under Grant
%%  No.~\grantnum{GS100000001}{nnnnnnn} and Grant
%%  No.~\grantnum{GS100000001}{mmmmmmm}.  Any opinions, findings, and
%%  conclusions or recommendations expressed in this material are those
%%  of the author and do not necessarily reflect the views of the
%%  National Science Foundation.
%%\end{acks}


%% Bibliography
\bibliography{refs.bib}


%% Appendix
%%\appendix
%%\section{Appendix}

%%Text of appendix \ldots

\end{document}
